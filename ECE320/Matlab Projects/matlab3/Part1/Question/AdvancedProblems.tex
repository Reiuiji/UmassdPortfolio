\begin{LARGE}
  \textbf{Advanced Problems}
\end{LARGE}

\hspace*{\fill}

Now consider the signal

\begin{center}
$x(t) = sin(\Omega_0 t + \frac{1}{2}\beta t^2)$.\\
\end{center}

which is often called a chirp signal due to the sound it makes when played through a loudspeaker. The "chirp" sound is due to the increasing instantaneous frequency of the signal over time. The instatansoue frequency of a sinusoidal signal is given by the derivatives of its phase, i.e., the argument of $sin(\cdot)$. For the chirp signal, the instantaneous frequency is

\[ 
  \begin{array}{c c l}
    \Omega_{inst}(t) & = & \text{$\frac{ d}{d x} \bigg( \Omega_0 t + \frac{1}{2} \beta t^2 \bigg)$} \\
     & = &  \text{$ \Omega_0 + \beta t$}
  \end{array} .\]

% \begin{array}{ccc}
%  \Omega_{inst}(t) & "=" & \frac{ d}{d x} \bigg( \Omega_0 t + \frac{1}{2} \beta t^2 \bigg) \\
%  = & \\
% \end{array}

Assume for the following problems that $\Omega_s = 2\pi(8192)$ rad/sec:

\begin{center}
[Note: Use $\beta = 2\pi(2000)$ for (g)-(j), not $\beta = 2000$ as in the book.]
\end{center}

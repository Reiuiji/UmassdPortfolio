
The modulation property for the DTFT can be used to construct other filters from a lowpass filter. The modulation property states that if $ X(e^{j(\omega - \omega_0)}) $. Use h and this property to design a real-values impulse response $h_1 [n]$ for a highpass filter whose frequency response magnitude approximates the ideal highpass filter:
\\
\[ H_{bsf}(e^{j \omega}) = \left\{
  \begin{array}{l l}
    1 & \quad \text{$4\pi/5 < |\omega| \le \pi$,} \\
    0 & \quad \text{otherwise.}
  \end{array} \right.\]
  

Store the impulse response of your new filter in h1. Generate an appropriately labelled plot of the magnitude of the frequency response for your highpass filter. Also, plot h1 using stem. How are h and h1 related?
